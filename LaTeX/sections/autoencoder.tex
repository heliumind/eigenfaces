\subsection{Autoencoder for feature extraction}
An Autoencoder is a neural network of multilayer structure that can learn
efficient coding of similar data, that allows reconstructing with an acceptable
accuracy. It consists of two parts, an encoder and a decoder. These overlap in
the middle of the network in the layer that will later contain the encoded data,
also called \textit{latent space} (see Fig. \ref{ae}). This layer has the function of a
``bottleneck'', it is significantly smaller than the input and output layer of
the autoencoder. Learning process is performed layer after layer. The
``bottleeck'' in the middle forces the encoder to abstract these vectors and
map them to a smaller representation. The decoder then tries to reconstruct the
original data. For feature extraction purposes one is merely interested in the
encoder part, which will be fed into the classification system.

Autoencoders are very well suited for extracting features from faces, because an
image of a face can be described much more space-savingly if you do not save the
RGB values of all pixels, but only the features that distinguish a certain face
from other faces. So, if the structure of the network and the size of the
"bottleneck" are cleverly chosen, the auto-encoder should automatically learn to
map the features of the faces in the code vectors. 

\begin{figure}[h]
  \centering
  \includegraphics[width=\columnwidth]{ae.png}
  \caption{Structure of an autoencoder}
  \label{ae}
\end{figure}

\subsection{Convolutional Neural Network}
As images are encoded in this project, I will be using an Autoencoder based on
Convolutional Neural Networks (CNN). This type of network is well suited for the
analysis of images, because in the first layers of the network small sections of
the present image are processed independently. Only in later layers are these
sections then merged together, so that the network can detect larger and more
complex structures of the image with each convolution layer without having to
look at the image in its entirety from the beginning. This saves a lot of
resources and achieves very good results in general. 

A CNN generally consists of three types of layers: \textit{Convolutions}, subsampling or
\textit{pooling layers}, and conventional \textit{fully-connected layers}.

Convolutions maintain the width and height of the three-dimensional neuron
matrix they receive, and only change the depth. They have a filter kernel that
applies any number of filters to any point of the obtained matrix. The result
for each filter is written to the output matrix at the same point, so that the
depth of the output matrix corresponds to the number of filters, but the width
and height remain unchanged.

Pooling layers reduce the width and height of the obtained matrix. They do this
by pooling adjacent points of the matrix. In \textit{Max-Pooling} for instance,
only the largest value for each 2×2 square is written to the output matrix.
Thus, the height and width of the matrix are halved, but the depth remains
unchanged. This has the purpose of discarding superfluous data so that later
convolution can recognize larger structures.

A CNN consists of a sequence of one or more convolutions, followed by a pooling
layer. If the size of the neuron matrix is sufficiently reduced, one or more
fully-connected layers follow in order to finally evaluate the data and
categorize it. Each layer is usually followed by the ReLU activation function. 
